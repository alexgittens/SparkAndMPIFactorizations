The differences in performance between the Cray{\textsuperscript{\tiny\textregistered}}~XC40{\textsuperscript{\tiny\texttrademark}} system~\cite{alverson2012cray,craycascadesc12} and the experimental Cray cluster point to optimizations to Spark that could improve its performance on HPC-style architectures.  The two platforms have very similar configurations, with the primary difference being the lack of local persistent storage on the XC40 nodes.  As described in Section~\ref{sect:h2h}, this forces some of Spark's local scratch space to be allocated on the remote Lustre file system, rather than in local storage.  To mitigate this, and keep more of the scratch data local, we propose the following future work:
\begin{itemize}
  \item Spark cleans its local scratch space inefficiently.
  In particular, shuffle data is not immediately cleaned up after a shuffle
  completes. This makes fault recovery more efficient, but results in higher
  storage requirements for scratch space. A more efficient cleaning process would
  make it more feasible to fit the scratch data entirely in a local RAM disk and
  avoid using Lustre.
\item Spark does not currently allow you to configure primary and backup
  scratch directories.  Instead you list all scratch directories in a single
  list, and it distributes data in a round round fashion between them as long
  as space is available.  You can bias it towards one storage device (e.g., RAM
  disk vs. Lustre) by listing multiple directories on the preferred device.
  Ideally, though, we would like to use a RAM disk (or other local storage)
  exclusively unless and until it fills, and only switch to Lustre directories
  if necessary.
\item Spark does not allow you to specify that a scratch directory is globally
  accessible.  Thus non-cached data is stored to the remote Lustre directory by
  the sender, and then later retrieved by the sender and sent to the receiver.
  This wastes a step, since the receiver could easily fetch the data directly
  from Lustre (or any other global file system).
\item Alternatively, a push model of communication (as opposed to the current
  pull model) might be possible - however this would have implications for
  reliability and handling of very large data sets.\footnote{Storing the
    shuffle data to a large persistent block storage device, and only sending
    it as needed, allows Spark to easily shuffle more data than could fit in
    the remote buffers.  In a push-based model, extra logic and synchronization
    would be necessary to ensure that the remote buffers do not overflow.}
\end{itemize}
